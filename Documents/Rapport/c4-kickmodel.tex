There are no principal changes to the kick model from soccer server
version 6 to soccer server version 7, so your old implementation
should still work. However, due to changes in the server parameter
file, in some cases multiple kicks are not necessary anymore. 

The \scommand{kick} command takes two parameters, the \param{kick
  power} the player client wants to use (between \sparam{minpower} and
\sparam{maxpower}) and the \param{angle} the player kicks the ball to.
The angle is given in degrees and has to be between \sparam{minmoment}
and \sparam{maxmoment} (see Tab.~\ref{tab:kickpar} for current
parameter values).

Once the
\scommand{kick} command arrived at the server, the kick will be
executed if the ball is kick-able for the player and the player is not
marked offside.  The ball is kick-able for the player, if the distance
between the player and the ball is between $0$ and
\sparam{kickable\_margin}. Heterogeneous players can have different
kickable margins. For the calculation of the distance during
this section, it is important to know that if we talk of distance 
between player and ball, we talk about the minimal distance between
the outer shape of both player and ball. So the distance in this
section is the distance between the center of both objects
\textbf{minus} the radius of the ball \textbf{minus} the radius of the
player.

The first thing to be calculated for the kick is the effective kick
power ep:
\begin{equation}
  \label{eq:effectivekick1}
  \mathrm{ep} = \mathrm{kick\ power} \cdot \mathrm{kick\_power\_rate}
\end{equation}

If the ball is not directly in front of the player, the effective kick
power will be reduced by a certain amount dependent on the position of
the ball with respect to the player. Both angle and distance are
important.

If the relative angle of the ball is 0{\textdegree} wrt.\ the body
direction of the player client --- i.e.\ the ball is in front of the
player --- the effective power will stay as it is. The larger the angle
gets, the more the effective power will be reduced. The worst case is
if the ball is lying behind the player (angle 180{\textdegree}): the
effective power is reduced by 25\%.

The second important variable for the effective kick power is the
distance from the ball to the player: it is quite obvious that --
should the kick be executed -- the distance between ball and player is
between $0$ and \sparam{kickable\_margin}. If the distance is 0, the
effective kick power will not be reduced again. The further the ball
is away from the player client, the more the effective kick power will
be reduced. If the ball distance is \sparam{kickable\_margin}, the
effective kick power will be reduced by 25\% of the original kick
power.

The overall worst case for kicking the ball is if a player kicks a
distant ball behind itself: 50\% of \param{kick power} will be used.
For the effective kick power, we get the formula
\ref{eq:effectivekick2}. (dir\_diff means the absolute direction
difference between ball and the player's body direction, dist\_diff
means the absolute distance between ball
and player.)\\
$0 \leq \mathrm{dir\_diff} \leq 180^\circ \quad\land\quad 0 \leq
\mathrm{dist\_diff} \leq \mathrm{kickable\_margin}$:
\begin{equation}
  \label{eq:effectivekick2}
  \mathrm{ep} = \mathrm{ep} \cdot \left(1- 0.25 \cdot \frac{\mathrm{dir\_diff}}{180^\circ} - 0.25 \cdot \frac{\mathrm{dist\_ball}}{\mathrm{kickable\_margin}} \right) 
\end{equation}


The effective kick power is used to calculate $\vec{a}_{{n}_{i}}$, an
acceleration vector that will be added to the global ball acceleration
$\vec{a}_{n}$ during cycle $n$ (remember that we have a multi agent
system and \emph{each} player close to the ball can kick it during the
same cycle).

There is a server parameter, \sparam{kick\_rand}, that can be used to
generate some noise to the ball acceleration. For the default players,
\sparam{kick\_rand} is 0 and no noise will be generated. For
heterogeneous players, \sparam{kick\_rand} depends on
\sparam{kick\_rand\_delta\_factor} in 
\file{player.conf} and on the actual kickable margin. In RoboCup 2000,
\sparam{kick\_rand} was used to generate some noise during evaluation
round for the normal players.

During the transition from simulation step $n$ to simulation step
$n+1$ acceleration $\vec{a}_{n}$ is applied: 

\begin{enumerate}
\item $\vec{a}_{n}$ is normalized to a maximum length of
  \sparam{baccel\_max}. Currently (Server 7), the maximum acceleration
  is equal to the maximum effective kick power. 
\item $\vec{a}_{n}$ is added to the current ball speed $\vec{v}_{n}$.
  $\vec{v}_{n}$ will be normalized to a maximum length of
  \sparam{ball\_speed\_max}.
\item Noise $\vec{n}$ and wind $\vec{w}$ will be added to
  $\vec{v}_{n}$. Both noise and wind are configurable in
  \file{server.conf}. Parameters responsible for the wind are
  \sparam{wind\_force}, \sparam{wind\_dir} and
  \sparam{wind\_rand}. The responsible parameter for the noise
  is \sparam{ball\_rand}. Both direction and length
  of the noise vector are within the interval $[ -|\vec{v}_{n}| \cdot
  \mathrm{ball\_rand} \ldots |\vec{v}_{n}| \cdot \mathrm{ball\_rand}]$.  
\item The new position of the ball $\vec{p}_{n+1}$ is the old position
  $\vec{p}_{n}$ plus the velocity vector $\vec{v}_{n}$ (i.e.\ the maximum
  distance difference for the ball between two simulation steps is 
  \sparam{ball\_speed\_max}).
\item \sparam{ball\_decay} is applied for the velocity of the ball:
  $\vec{v}_{n+1} = \vec{v}_{n} \cdot \mathrm{ball\_decay}$.
  Acceleration $\vec{a}_{n+1}$ is set to zero.
\end{enumerate}

With the current settings the ball covers a distance up to 45, assuming
an optimal kick. 53 cycles after an optimal kick, the distance from the 
kick off position to the ball is about 43, the remaining velocity is smaller 
than $0.1$. 15 cycles after an optimal kick, the ball covers a distance of 
27 -- 28 and the remaining veloctity is slightly larger than 1. 

Implications from the kick model and the current parameter settings are 
that it still might be helpful to use several small kicks for a compound 
kick -- for example stopping the ball, kick it to a more advantageous 
position within the kickable area and kick it to the desired direction.
It would be another possibility to accelerate the ball to maximum speed
without putting it to relative position (0,0{\textdegree}) using a 
compound kick. 

\begin{table}[htbp]
  \begin{center}
    \begin{tabular}[h]{|l|r|l|r|r|}
      \hline
      \multicolumn{2}{|c|}{\textbf{Basic Parameters}} &
      \multicolumn{3}{c|}{\textbf{Parameters for heterogeneous
        Players}} \\ 
      \multicolumn{2}{|c|}{\file{server.conf}} &
      \multicolumn{3}{c|}{\file{player.conf}} \\ \hline
      \textbf{Name} & \textbf{Value} & \textbf{Name} & \textbf{Value} &
      \textbf{Range} \\ \hline
      \sparam{minpower} & -100 & & & \\ \hline
      \sparam{maxpower} & 100 & & & \\ \hline
      \sparam{minmoment} & -180 & & & \\ \hline
      \sparam{maxmoment} & 180 & & & \\ \hline
      \sparam{kickable\_margin} & 0.7 &
      \sparam{kickable\_margin\_delta\_min} & 0.0 & 0.7 \\
      & & \sparam{kickable\_margin\_delta\_max} & 0.2 & ---~0.9 \\ \hline
      \sparam{kick\_power\_rate} & 0.027 & & & \\ \hline
      \sparam{kick\_rand} & 0.0
      & \sparam{kick\_rand\_delta\_factor} & 0.5 & \\
      & & \sparam{kickable\_margin\_delta\_min} & 0.0 & \rb{1.5ex}{0.0} \\
      & & \sparam{kickable\_margin\_delta\_max} & 0.2
      & \rb{1.5ex}{---~0.1} \\\hline
      \sparam{ball\_size} & 0.085 & & & \\\hline
      \sparam{ball\_decay} & 0.94 & & & \\\hline
      \sparam{ball\_rand} & 0.05 & & & \\\hline
      \sparam{ball\_speed\_max} & 2.7 & & & \\\hline
      \sparam{ball\_accel\_max} & 2.7 & & & \\\hline     
      \sparam{wind\_force} & 0.0 & & & \\\hline
      \sparam{wind\_dir} & 0.0 & & & \\\hline
      \sparam{wind\_rand} & 0.0 & & & \\\hline            
    \end{tabular}
    \caption{Ball and Kick Model Parameters}
    \label{tab:kickpar}
  \end{center}
\end{table}


%%% Local Variables: 
%%% mode: latex
%%% TeX-master: "manual"
%%% End: 
